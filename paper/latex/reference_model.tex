In this work, the chosen reference model is
the following linear heat equation with convection
boundary conditions: solve for the temperature
field $T = f(x,t)$ such that

\begin{align}
  \rho c_p \partial_t T{\scriptstyle(x,t)} - \nabla \cdot (k \nabla T{\scriptstyle(x,t)}) &= r{\scriptstyle(x,t)}
  \qquad &\forall x &\in \Omega \label{eq:refheateq}\\
  q_{conv} &= h_{conv} ( T{\scriptstyle(x,t)} - T_{\textrm{env}} ) \qquad & \forall x &\in \partial \Omega {}_{conv}\notag
\end{align}

, where $\rho, \enskip c_p, \enskip k, \enskip  r, \enskip q_{conv} \textrm{ and } h_{conv}$
denote respectively
the density, the specific heat, the conductivity, volumetric
heat source, the convection heat flux and the convection coefficient.
The heat source adopted in this work is a standard Gaussian source:

\begin{equation}\label{eq:heatsource}
  r{\scriptstyle(\mathbf{x}, t)} = \frac{6 \sqrt{3} P}{ \pi^{3/2} R^3}
  \exp\bigg( \frac{-3||\mathbf{x} - \mathbf{x_{hs}}||_2^2}{R^2}\bigg)
\end{equation}

, where $P, \enskip R \textrm{ and } \mathbf{x_{hs}}$
denote respectively the power, the radius and the position
of the heat source.\par

The time dependency of the domain is due to
the deposition
\footnote{ In the case of LPBF, this means that the powder is not modelled. }.
Numerically, this is modelled using  the born-dead element
technique \citep{Chiumenti2010}:
elements that are \textit{inactive} are not included in the
computational domain nor assembled,
and, at the beginning of each time step, a collision test
is performed with the geometry of the deposition
in order to \textit{activate} elements corresponding to
the newly deposited material.\par

Although this model misses important features like
thermal dependency of the parameters, radiation heat loss and
latent heat\citep{VanElsen2007, Hodge2021},
it has the advantage of being the simplest setting
where the proposed method can be illustrated.\par
