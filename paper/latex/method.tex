\subsection{Brief description of reference model}

In this work, the chosen reference model is
the following linear heat equation with convection
boundary conditions: solve for the temperature
field $T = f(x,t)$ such that

\begin{align}
  \rho c_p \partial_t T{\scriptstyle(x,t)} - \nabla \cdot (k \nabla T{\scriptstyle(x,t)}) &= r{\scriptstyle(x,t)}
  \qquad &\forall x &\in \Omega \label{eq:refheateq}\\
  q_{conv} &= h_{conv} ( T{\scriptstyle(x,t)} - T_{env} ) \qquad & \forall x &\in \partial \Omega {}_{conv}\notag
\end{align}

, where $\rho, \enskip c_p, \enskip k, \enskip  r, \enskip q_{conv} \textrm{ and } h_{conv}$
denote respectively
the density, the specific heat, the conductivity, volumetric
heat source, the convection heat flux and the convection coefficient.
The heat source adopted in this work is a standard Gaussian source:

\begin{equation}\label{eq:heatsource}
  r(x, t) = \frac{6 \sqrt{3} P}{ \pi^{3/2} R^3}
  \exp\bigg( \frac{-3||\mathbf{x} - \mathbf{x_{hs}}||_2^2}{R^2}\bigg)
\end{equation}

, where $P, \enskip R \textrm{ and } \mathbf{x_{hs}}$
denote respectively the power, the radius and the position
of the heat source.\par

The time dependency of the domain means, in the particular case
of LPBF, 
that the powder bed is not included.
Numerically, the deposition of material
is modelled using  the born-dead element
technique \cite{Chiumenti2010}:
elements that are \textit{inactive} are not included in the
computational domain,
and, at the beginning of each time step, a collision test
is performed with the geometry of the deposition
in order to \textit{activate} elements.\par

Although missing important features like
thermal dependency of the parameters, radiation heat loss and
latent heat\cite{VanElsen2007, Hodge2021}, this problem
has the advantage of being the simplest setting where
the proposed method can be illustrated.\par

The PDE \ref{eq:refheateq} is discretized in space using P1/Q1
finite elements and in time using Backward Euler time integration
\cite{Puso2023}.\par

\subsection{Proposed model}

Let us denote the change of reference frame by

$$
\mathbf{\xi} = \mathbf{x} - \int_0^t \mathbf{V}(t) dt
$$

, where $\mathbf{V} = f(t)$ denotes the
speed of the laser.
The proposed model is as follows:
in a \textit{moving} subdomain containing the heat source,
$\Omega_m \subset \Omega$
the problem is posed in its reference frame.
In the remaining
\textit{fixed} subdomain $\Omega_f \subset \Omega$,
the problem is posed
in the usual laboratory reference frame.
The chosen domain decomposition framework is
a standard Neumann Dirichlet non-overlapping decomposition,
hence $\Omega_f = (\Omega \setminus \Omega_m)^{\mathrm{o}}$,
where ${}^{\mathrm{o}}$ denotes the interior of a set. Let
$\Gamma$ denote the interface between both subdomains,
$\Gamma = \partial \Omega_m \cap \partial \Omega_f$. In this
setting, the system \ref{eq:refheateq} now writes 

\begin{align}
  \rho c_p \Big( \partial_t T_m{\scriptstyle (\xi, t)} - \mathbf{V} \cdot \nabla T_m{\scriptstyle (\xi, t)} \Big) -
  \nabla \cdot ( k \nabla T_m{\scriptstyle (\xi, t)}) &= r{\scriptstyle (\xi)}  &\xi &\in \Omega_m \label{eq:pdeddm} \\
  k \partial_n T_m{\scriptstyle (\xi, t)} &= k \partial_n T_f  &\xi &\in \Gamma \notag\\
  q_{\textrm{conv}} &= h_{\textrm{conv}} ( T_m{\scriptstyle(\xi,t)} - T_{env} ) \qquad &\xi &\in \partial \Omega_{m, \textrm{conv}}\notag\\
  \rho c_p \partial_t T_f{\scriptstyle (x, t)} - \nabla \cdot (k \nabla T_f{\scriptstyle (x, t)}) &= r{\scriptstyle (x, t)} \qquad &x &\in \Omega_f \label{eq:pdeddf} \\
  T_f{\scriptstyle (x, t)} &= T_m \qquad &x &\in \Gamma \notag\\
  q_{\textrm{conv}} &= h_{\textrm{conv}} ( T_f{\scriptstyle(x,t)} - T_{env} ) \qquad &x &\in \partial \Omega_{f, \textrm{conv}}\notag
\end{align}
, where $\mathbf{V}$ denotes the speed of the heat source.

\begin{align}
  \int v_m \rho c_p \partial_t T_m 
  - \int v_m \rho c_p \mathbf{V} \cdot \nabla T_m
  + \int k \nabla v_m \cdot \nabla T_m
  = \int v_m r
\end{align}

\todo{Finish weak forms}\par
%As for the reference method, PDEs \ref{eq:pdeddm} and
%\ref{eq:pdeddf} are discretized in space using P1/Q1 finite
%elements and in time using Backward Euler.

\todo{Talk about geometrical operations}\par
\todo{Overview of timestep}\par

\begin{figure}
  \castelincfig{nonOverlappingPartition}
  \caption{Schematic of proposed model.}
  \label{fig:schematic}
\end{figure}


\iffalse
Show reference model
Go over limitations
Introduce my model
\fi
