\subsection{Brief description of reference model}

In this work, the chosen reference model is
the following linear heat equation with convection
boundary conditions: solve for the temperature
field $T = f(x,t)$ such that

\begin{align}
  \rho c_p \partial_t T{\scriptstyle(x,t)} - \nabla \cdot (k \nabla T{\scriptstyle(x,t)}) &= r{\scriptstyle(x,t)}
  \qquad &\forall x &\in \Omega \label{eq:refheateq}\\
  q_{conv} &= h_{conv} ( T{\scriptstyle(x,t)} - T_{env} ) \qquad & \forall x &\in \partial \Omega {}_{conv}
\end{align}

, where $\rho, \enskip c_p, \enskip k \textrm{ and } r$
denote respectively
the density, the specific heat, the conductivity and the volumetric
heat source. In the boundary condition,
$q_{conv}, \enskip h_{conv}$ denote the convective heat flux
and coefficient.
This is the simplest setting where
the proposed method can be illustrated.
Notably, the thermal dependency of the parameters
and the latent heat are being neglected. The latent heat is
arguably relevant in LPBF applications
\cite{VanElsen2007, Hodge2021}.\par

The time dependency of the domain means, in the particular case
of LPBF, 
that the powder bed is not included.
Numerically, this is modelled
using  the born-dead element technique \cite{Chiumenti2010}:
elements that are \textit{inactive} are not included in the
computational domain,
and, at the beginning of each time step, a collision test
is performed with the geometry of the deposition
in order to \textit{activate} elements.\par

The PDE \ref{eq:refheateq} is discretized in space using P1/Q1
finite elements and in time using Backward Euler time integration
\cite{Puso2023}.\par

\subsection{Proposed model}

\todo{The model we propose is as follows}

\begin{align}
  \label{eq:pdeddf}
  \rho c_p \partial_t T_f - \nabla \cdot (k \nabla T_f) &= r(x - Vt) \qquad &x &\in \Omega_f\\
  T_f &= T_m \qquad &x &\in \Gamma \notag
\end{align}
\begin{align}
  \label{eq:pdeddm}
  \rho c_p \Big( \partial_t T_m - \mathbf{V} \cdot \nabla T_m \Big) -
  \nabla \cdot ( k \nabla T_m) &= r(\xi)  &\xi &\in \Omega_m\\
  k \partial_n T_m &= k \partial_n T_f  &\xi &\in \Gamma \notag
\end{align}

\begin{figure}
  \castelincfig{nonOverlappingPartition}
  \caption{Schematic of proposed model.}
  \label{fig:schematic}
\end{figure}


\iffalse
Show reference model
Go over limitations
Introduce my model
\fi
