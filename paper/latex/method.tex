\subsection{Brief description of reference model}
  In this work, the chosen reference model is
the following linear heat equation with convection
boundary conditions: solve for the temperature
field $T = f(x,t)$ such that

\begin{align}
  \rho c_p \partial_t T{\scriptstyle(x,t)} - \nabla \cdot (k \nabla T{\scriptstyle(x,t)}) &= r{\scriptstyle(x,t)}
  \qquad &\forall x &\in \Omega \label{eq:refheateq}\\
  q_{conv} &= h_{conv} ( T{\scriptstyle(x,t)} - T_{env} ) \qquad & \forall x &\in \partial \Omega {}_{conv}\notag
\end{align}

, where $\rho, \enskip c_p, \enskip k, \enskip  r, \enskip q_{conv} \textrm{ and } h_{conv}$
denote respectively
the density, the specific heat, the conductivity, volumetric
heat source, the convection heat flux and the convection coefficient.
The heat source adopted in this work is a standard Gaussian source:

\begin{equation}\label{eq:heatsource}
  r(\mathbf{x}, t) = \frac{6 \sqrt{3} P}{ \pi^{3/2} R^3}
  \exp\bigg( \frac{-3||\mathbf{x} - \mathbf{x_{hs}}||_2^2}{R^2}\bigg)
\end{equation}

, where $P, \enskip R \textrm{ and } \mathbf{x_{hs}}$
denote respectively the power, the radius and the position
of the heat source.\par

The time dependency of the domain is due to
the deposition
\footnote{ In the case of LPBF, this means that the powder is not modelled. }.
Numerically, this is modelled using  the born-dead element
technique \citep{Chiumenti2010}:
elements that are \textit{inactive} are not included in the
computational domain nor assembled,
and, at the beginning of each time step, a collision test
is performed with the geometry of the deposition
in order to \textit{activate} elements corresponding to
the newly deposited material.\par

Although this model misses important features like
thermal dependency of the parameters, radiation heat loss and
latent heat\citep{VanElsen2007, Hodge2021},
it has the advantage of being the simplest setting
where the proposed method can be illustrated.\par


\subsection{Proposed model}

Let us denote the change of reference frame by

$$
\mathbf{\xi} = \mathbf{x} - \int_0^t \mathbf{V}(t) dt
$$

, where $\mathbf{V} = f(t)$ denotes the
speed of the laser.
The proposed model is as follows:
in a \textit{moving} subdomain attached to the heat source,
$\Omega_m \subset \Omega$,
the problem is posed in the reference frame of the heat source.
In the remaining
\textit{fixed} subdomain $\Omega_f \subset \Omega$,
the problem is posed
in the usual laboratory reference frame.
The chosen domain decomposition framework is
a standard Neumann Dirichlet non-overlapping decomposition,
hence $\Omega_f = (\Omega \setminus \Omega_m)^{\mathrm{o}}$,
where ${}^{\mathrm{o}}$ denotes the interior of a set. Let
$\Gamma$ denote the interface between both subdomains,
$\Gamma = \partial \Omega_m \cap \partial \Omega_f$. In this
setting, system \ref{eq:refheateq} can be rewritten as
a coupled PDE system

\begin{align}
  \rho c_p \Big( \partial_t T_m{\scriptstyle (\xi, t)} - \mathbf{V} \cdot \nabla T_m{\scriptstyle (\xi, t)} \Big) -
  \nabla \cdot ( k \nabla T_m{\scriptstyle (\xi, t)}) &= r{\scriptstyle (\xi)}  &\xi &\in \Omega_m \label{eq:pdeddm}\\
  k \partial_n T_m{\scriptstyle (\xi, t)} &= k \partial_n T_f  &\xi &\in \Gamma \notag\\
  q_{\textrm{conv}} &= h_{\textrm{conv}} ( T_m{\scriptstyle(\xi,t)} - T_{\textrm{env}} ) \qquad &\xi &\in \partial \Omega_{m, \textrm{conv}}\notag\\
  \rho c_p \partial_t T_f{\scriptstyle (x, t)} - \nabla \cdot (k \nabla T_f{\scriptstyle (x, t)}) &= r{\scriptstyle (x, t)} \qquad &x &\in \Omega_f \label{eq:pdedf}\\
  T_f{\scriptstyle (x, t)} &= T_m \qquad &x &\in \Gamma \label{eq:dirichletdf}\\
  q_{\textrm{conv}} &= h_{\textrm{conv}} ( T_f{\scriptstyle(x,t)} - T_{\textrm{env}} ) \qquad &x &\in \partial \Omega_{f, \textrm{conv}}\notag
\end{align}
Note that
the time dependency of heat source is absent from \ref{eq:pdeddm}
Ideally, $\Omega_m$ is big enough
to contain the whole support of the heat source such that
the right hand side of \ref{eq:pdedf} is null.
For both this model and the reference, the spatial discretization
is carried out using P1/Q1 finite elements and Backward Euler
time integration. \citep{Puso2023} mentions that higher order
time integrations will introduce further oscillations.
\textit{Streamline Upwind Petrov-Galerkin} (SUPG) stabilization is chosen
in order to take care of the newly introduced advection term
in \ref{eq:pdeddm}. The corresponding semi-discrete weak forms write

\begin{align}
  \label{eq:weakformdm}
    \int_{\Omega_m} v_m \rho c_p \partial_t \overline{T}_m 
  - \int_{\Omega_m} v_m \rho c_p \mathbf{V} \cdot \nabla \overline{T}_m
  + \int_{\Omega_m} k \nabla v_m \cdot \nabla \overline{T}_m&\\
  - \int_{\partial \Omega_{m, \textrm{conv}}} \hspace{-8mm} v_m \mathbf{q_{conv}} \cdot \hat{n}
  - \int_{\Gamma} k v_m \nabla \overline{T}_f \cdot \hat{n} &\notag\\
  + \int_{\Omega_m} \tau R_h \rho \mathbf{V} \cdot \nabla v_m &= \int_{\Omega_m} v_m r \notag\\
  \label{eq:weakformdf}
    \int_{\Omega_f} v_f \rho c_p \partial_t \overline{T}_f 
  + \int_{\Omega_f} k \nabla v_f \cdot \nabla \overline{T}_f
  - \int_{\partial \Omega_{f, \textrm{conv}}} \hspace{-8mm} v_f \mathbf{q_{conv}} \cdot \hat{n} &= \int_{\Omega_f} v_f r
\end{align}

, where $R_h$ denotes the residual of equation \ref{eq:pdeddm}

$$
R_h = \rho c_p \partial_t \overline{T}_m + \rho c_p \mathbf{V} \cdot \nabla \overline{T}_m - \nabla \cdot ( k \nabla \overline{T}_m) - r
$$

% EXPLAIN MESHING
Both problems on $\Omega_m$ and $\Omega_f$
are meshed separately, hence if both meshes
have the same mesh size the interface nodes are duplicated.
The described coupled problem is solved monolitically.
Note that, since the Dirichlet values of boundary condition
\ref{eq:dirichletdf} are not known before solving,
they have to be imposed at the algebraic level.
The resulting linear system writes

\begin{equation}\label{eq:algebraic_coupled}
  \begin{pmatrix}
    \frac{\tilde{M}_m}{\Delta t} + A_{m, m} & A_{m, m_\Gamma} & \cdot & \cdot \\[3mm]
    A_{m_\Gamma, m} & \frac{\tilde{M}_{m_\Gamma}}{\Delta t} + A_{m_\Gamma, m_\Gamma} & \cdot & A_{m_\Gamma, f_\Gamma}  \\[3mm]
    \cdot & A_{f, m_\Gamma} & \frac{\tilde{M}_{f}}{\Delta t}  + A_{f, f} & A_{f, f_\Gamma}  \\[3mm]
    \cdot & C & \cdot & -I
  \end{pmatrix}
  \begin{pmatrix}
    \overline{T}_m^{n+1}          \\[3mm]
    \overline{T}_{m_\Gamma}^{n+1} \\[3mm]
    \overline{T}_f^{n+1}          \\[3mm]
    \overline{T}_{f_\Gamma}^{n+1}
  \end{pmatrix}
  =
  \begin{pmatrix}
    F_m          + \frac{\tilde{M}_m}{\Delta t} \overline{T}_m^{n}\\[3mm]
    F_{m_\Gamma} + \frac{\tilde{M}_{m_\Gamma}}{\Delta t} \overline{T}_{m_\Gamma}^{n}\\[3mm]
    F_{f}        + \frac{\tilde{M}_{f}}{\Delta t} \overline{T}_{f}^{n}\\[3mm]
    \cdot
  \end{pmatrix}
\end{equation}
, where $\tilde{M}$ matrices denote modified mass matrices with an extra SUPG term,
, the A matrices englobe advection, diffusion, Neumann and convection contributions
where each of them apply and $C$ is a matrix whose rows add up to 1.\par

\begin{figure}
  \castelincfig{nonOverlappingPartition}
  \caption{Schematic of proposed model.}
  \label{fig:schematic}
\end{figure}

\todo{Talk about geometrical operations}\par
\todo{Overview of timestep}\par
\todo{
  \begin{enumerate}
    \item Why this model: more steady
    \item Deciding $\Delta t$
    \item Meshing of subdomain and deciding size/shape of subdomain
  \end{enumerate}
}

\iffalse
Show reference model
Go over limitations
Introduce my model
\fi
