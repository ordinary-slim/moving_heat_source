\subsection{Brief description of reference model}
  In this work, the chosen reference model is
the following linear heat equation with convection
boundary conditions: solve for the temperature
field $T = f(x,t)$ such that

\begin{align}
  \rho c_p \partial_t T{\scriptstyle(x,t)} - \nabla \cdot (k \nabla T{\scriptstyle(x,t)}) &= r{\scriptstyle(x,t)}
  \qquad &\forall x &\in \Omega \label{eq:refheateq}\\
  q_{conv} &= h_{conv} ( T{\scriptstyle(x,t)} - T_{env} ) \qquad & \forall x &\in \partial \Omega {}_{conv}\notag
\end{align}

, where $\rho, \enskip c_p, \enskip k, \enskip  r, \enskip q_{conv} \textrm{ and } h_{conv}$
denote respectively
the density, the specific heat, the conductivity, volumetric
heat source, the convection heat flux and the convection coefficient.
The heat source adopted in this work is a standard Gaussian source:

\begin{equation}\label{eq:heatsource}
  r(\mathbf{x}, t) = \frac{6 \sqrt{3} P}{ \pi^{3/2} R^3}
  \exp\bigg( \frac{-3||\mathbf{x} - \mathbf{x_{hs}}||_2^2}{R^2}\bigg)
\end{equation}

, where $P, \enskip R \textrm{ and } \mathbf{x_{hs}}$
denote respectively the power, the radius and the position
of the heat source.\par

The time dependency of the domain is due to
the deposition
\footnote{ In the case of LPBF, this means that the powder is not modelled. }.
Numerically, this is modelled using  the born-dead element
technique \citep{Chiumenti2010}:
elements that are \textit{inactive} are not included in the
computational domain nor assembled,
and, at the beginning of each time step, a collision test
is performed with the geometry of the deposition
in order to \textit{activate} elements corresponding to
the newly deposited material.\par

Although this model misses important features like
thermal dependency of the parameters, radiation heat loss and
latent heat\citep{VanElsen2007, Hodge2021},
it has the advantage of being the simplest setting
where the proposed method can be illustrated.\par


\subsection{Proposed model}

Let us denote the change of reference frame by

$$
\mathbf{\xi} = \mathbf{x} - \int_0^t \mathbf{V}(t) dt
$$

, where $\mathbf{V} = f(t)$ denotes the
speed of the laser.
The proposed model is as follows:
in a \textit{moving} subdomain attached to the heat source,
$\Omega_m \subset \Omega$,
the problem is posed in the reference frame of the heat source.
In the remaining
\textit{fixed} subdomain $\Omega_f \subset \Omega$,
the problem is posed
in the usual laboratory reference frame.
The chosen domain decomposition framework is
a standard Neumann Dirichlet non-overlapping decomposition,
hence $\Omega_f = (\Omega \setminus \Omega_m)^{\mathrm{o}}$,
where ${}^{\mathrm{o}}$ denotes the interior of a set. Let
$\Gamma$ denote the interface between both subdomains,
$\Gamma = \partial \Omega_m \cap \partial \Omega_f$. In this
setting, system \ref{eq:refheateq} can be rewritten as

\begin{align}
  \rho c_p \Big( \partial_t T_m{\scriptstyle (\xi, t)} - \mathbf{V} \cdot \nabla T_m{\scriptstyle (\xi, t)} \Big) -
  \nabla \cdot ( k \nabla T_m{\scriptstyle (\xi, t)}) &= r{\scriptstyle (\xi)}  &\xi &\in \Omega_m \label{eq:pdeddm}\\
  k \partial_n T_m{\scriptstyle (\xi, t)} &= k \partial_n T_f  &\xi &\in \Gamma \notag\\
  q_{\textrm{conv}} &= h_{\textrm{conv}} ( T_m{\scriptstyle(\xi,t)} - T_{env} ) \qquad &\xi &\in \partial \Omega_{m, \textrm{conv}}\notag\\
  \rho c_p \partial_t T_f{\scriptstyle (x, t)} - \nabla \cdot (k \nabla T_f{\scriptstyle (x, t)}) &= r{\scriptstyle (x, t)} \qquad &x &\in \Omega_f \label{eq:pdedf}\\
  T_f{\scriptstyle (x, t)} &= T_m \qquad &x &\in \Gamma \notag\\
  q_{\textrm{conv}} &= h_{\textrm{conv}} ( T_f{\scriptstyle(x,t)} - T_{env} ) \qquad &x &\in \partial \Omega_{f, \textrm{conv}}\notag
\end{align}
, where $\mathbf{V}$ denotes the speed of the heat source. Note that
the time dependency of heat source is absent from \ref{eq:pdeddm}
Ideally, $\Omega_m$ is big enough
to contain the whole support of the heat source such that
the right hand side of \ref{eq:pdedf} is null.
\textit{Steamline Upwind Petrov Galerkin} stabilization is chosen
in order to take care of the newly introduced advection term
in \ref{eq:pdeddm}. The corresponding weak forms write respectively

\begin{align}
  \label{eq:weakformdm}
    \int_{\Omega_m} v_m \rho c_p \partial_t T_m 
  - \int_{\Omega_m} v_m \rho c_p \mathbf{V} \cdot \nabla T_m
  + \int_{\Omega_m} k \nabla v_m \cdot \nabla T_m&\\
  - \int_{\partial \Omega_{m, \textrm{conv}}} \hspace{-8mm} v_m \mathbf{q_{conv}} \cdot \hat{n}
  - \int_{\Gamma} k v_m \nabla T_f \cdot \hat{n} &\notag\\
  + \int_{\Omega_m} \tau r_h  \rho \mathbf{V} \cdot \nabla v_m &= \int_{\Omega_m} v_m r \notag\\
  \label{eq:weakformdf}
    \int_{\Omega_f} v_f \rho c_p \partial_t T_f 
  + \int_{\Omega_f} k \nabla v_f \cdot \nabla T_f
  - \int_{\partial \Omega_{f, \textrm{conv}}} \hspace{-8mm} v_f \mathbf{q_{conv}} \cdot \hat{n} &= \int_{\Omega_f} v_f r
\end{align}

, where $r_h$ is the residual of equation \ref{eq:pdeddm}.
For both this model and the reference, the spatial discretization
is carried out using P1/Q1 finite elements and Backward Euler
time integration. \cite{Puso2023} mentions that higher order
time integrations will introduce further oscillations.\par

% EXPLAIN MESHING
After discretization, the algebraic form obtained from
\ref{eq:weakformdm} and \ref{eq:weakformdf} writes as follows

\begin{equation}\label{eq:algebraic_coupled}
  \begin{bmatrix}
    \tilde{M} & \cdot \\
    \cdot & \tilde{M}
  \end{bmatrix}
\end{equation}

% EXPLAIN ASSEMBLY
\todo{Talk about geometrical operations}\par
\todo{Overview of timestep}\par

\begin{figure}
  \castelincfig{nonOverlappingPartition}
  \caption{Schematic of proposed model.}
  \label{fig:schematic}
\end{figure}


\iffalse
Show reference model
Go over limitations
Introduce my model
\fi
