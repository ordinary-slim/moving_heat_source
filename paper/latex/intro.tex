\todo{Additive manufacturing is cool}\par
\todo{However, simulation is slower than experiment}\par
\todo{Simulation is difficult because
is ``extremely multiscale'' \cite{Hodge2021}}\par
\todo{Time scale of radius and constraint on timestep}\par
\todo{Special techniques are required
\cite{Puso2023, Viguerie2022}}\par
\todo{Motivate new technique \cite{VanElsen2007, Mundra1996, Powar2016, Storti2022}}\par

\begin{table}[h]
  \centering
  \begin{tabular}{|c|c|c|}
    \hline
    & DED & LPBF \\
    \hline
    $L_{print}$ & $\mathcal{O}(1 m   )$ &  $\mathcal{O}(100 mm    )$  \\
    $L_{hs}$    & $\mathcal{O}(1 mm  )$ &  $\mathcal{O}(100 \mu m)$  \\
    $T_{print}$ & $\mathcal{O}(1 h   )$ &  $\mathcal{O}(1 h       )$  \\
    $T_{hs}$    & $\mathcal{O}(0.1 s )$ &  $\mathcal{O}(0.001 s  )$  \\
    \hline
  \end{tabular}
  \caption{Estimates for length and time scales involved in metallic AM. The time
  scale of the heat source is obtained by dividing its radius by its
  speed.}
  \label{tbl:scales}
\end{table}

Simulation is expensive due to the ``extremely multiscale'' nature
of the problem \cite{Hodge2021}. Table \ref{tbl:scales}
gives some magnitudes for the different scales involved in the AM process.
The time scale of the heat source manifests at the discrete level
as a constraint on the maximum timestep. It is useful to introduce
as a unit of time the time it takes the heat source to travel a radius.

\begin{equation}\label{eq:trad}
\mathcal{R} = \frac{R}{V}
\end{equation}

For $\Delta t \geq 1 \mathcal{R}$, elements are skipped over resulting
in a spotty pattern of the thermal field.
\cite{Hodge2021} estimates that the simulation of a
$19.2 \times 19.2 \times 19.2 mm^3$ cube LPBF print would take 
131 million timesteps with $\Delta t = 1 \mathcal{R}$ and
874 million timesteps with $\Delta t = 0.15 \mathcal{R}$.
Many authors opt for
$\Delta t = 0.5\mathcal{R}$ \cite{Patil2021, Michaleris2014, Stender2018}
and commercial softwares like \texttt{Netfabb}
do not allow for a time-step bigger than $1 \mathcal{R}$
\footnote{\url{https://help.autodesk.com/view/NETF/2021/ENU/?guid=GUID-67A6D8F0-AA04-436B-9FAF-A912BE00B497}}.
This constraint being too restrictive, two sets of solutions
appear in the litterature:

\begin{itemize}
  \item \textit{Lumped heat source}: Averaged source terms
    like the ones proposed in \cite{Chiumenti2017, Malmelv2019}
    allow one to bypass the restriction of
    the heat source's timescale and time-step hatches
    or even layers at a time at the cost of accuracy.
    \cite{VanElsen2007} shows that although the
    isothermals far from the heat source coincide,
    the HAZ volume and the peak temperature are
    greatly mispredicted.
  \item \textit{Spatially partitioned time evolution}: Methods
    that allow for finer timesteps close to the heat source
    and coarser timesteps far from it. \cite{Kopp2022} and
    \cite{Viguerie2022} implement this for welding problems
    through space time FEM and domain decomposition methods respectively.
    \cite{Viguerie2022} claims a computational speedup of 2.44.
    \cite{Puso2023} claims a speedup of at least a factor of 20
    using the multirate time integration described in \cite{Hodge2021}.
\end{itemize}
