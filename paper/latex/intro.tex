In the last 30 years, Additive Manufacturing (AM) of metals has transitioned from
prototyping to a production technology. The intense and localized nature of the
heat source generates thermal gradients which in turn produce residual stresses
and part distorsion \cite{Fu2014}.
These mechanisms may throw the product out of the threshold defined by
the required geometrical and mechanical tolerances.
The computational simulation of the process can help
scientists, engineers and end users further understand these mechanisms
and find ways to reduce their impact \cite{Lu2022}.\par

\begin{table}[h]
  \centering
  \begin{tabular}{|c|c|c|}
    \hline
    & DED & LPBF \\
    \hline
    $L_{print}$ & $\mathcal{O}(1 m   )$ &  $\mathcal{O}(100 mm    )$  \\
    $L_{hs}$    & $\mathcal{O}(1 mm  )$ &  $\mathcal{O}(100 \mu m)$  \\
    $T_{print}$ & $\mathcal{O}(1 h   )$ &  $\mathcal{O}(1 h       )$  \\
    $T_{hs}$    & $\mathcal{O}(0.1 s )$ &  $\mathcal{O}(0.0001-0.001 s  )$  \\
    \hline
  \end{tabular}
  \caption{Estimates for length and time scales involved in metallic AM. The time
  scale of the heat source is obtained by dividing its radius by its
  speed.}
  \label{tbl:scales}
\end{table}

However, simulation of AM processes and in particular Laser Powder Bed Fusion (LPBF), is
expensive due to the ``extremely multiscale'' nature
of the problem \cite{Hodge2021}.
Table \ref{tbl:scales}
gives some estimates for the different scales involved in Directed Energy Deposition (DED) and LPBF.
The time scale of the heat source, obtained from its length scale and its speed,
is prohibitively small in comparison to the total print time.
Said scale manifests at the discrete level
as a constraint on the maximum timestep; in this context
it is useful to introduce as a unit
the time it takes the heat source to travel a radius

\begin{equation}\label{eq:trad}
\mathcal{R} = \frac{R}{V}
\end{equation}

Then, for $\Delta t \geq 1 \mathcal{R}$, elements are skipped over resulting
in a spotty pattern of the thermal field.
For this reason, many authors opt for
$\Delta t = 0.5\mathcal{R}$ \cite{Patil2021, Michaleris2014, Stender2018}
and commercial softwares like \texttt{Netfabb}
do not allow for a time-step bigger than $1 \mathcal{R}$
\footnote{\url{https://help.autodesk.com/view/NETF/2021/ENU/?guid=GUID-67A6D8F0-AA04-436B-9FAF-A912BE00B497}}.
\cite{Hodge2021} estimates that the simulation of a
$19.2 \times 19.2 \times 19.2 mm^3$ cube LPBF print would take 
131 million timesteps with $\Delta t = 1 \mathcal{R}$ and
874 million timesteps with $\Delta t = 0.15 \mathcal{R}$.
This constraint being too restrictive, two sets of solutions
appear in the litterature:

\begin{itemize}
  \item \textit{Lumped heat source}: Replacing the accurate
    heat source profile by averaged source terms
    like the ones proposed in \cite{Chiumenti2017, Malmelv2019}
    allows one to bypass the restriction of
    the heat source's timescale and time-step hatches
    or even layers at a time at the cost of accuracy.
    \cite{VanElsen2007} shows that although the
    isothermals far from the heat source coincide,
    the HAZ volume and the peak temperature are
    greatly mispredicted.
  \item \textit{Spatially partitioned time evolution}: Methods
    that allow for finer timesteps close to the heat source
    and coarser timesteps far from it. \cite{Kopp2022} and
    \cite{Viguerie2022} implement this for welding problems
    through space time FEM and domain decomposition methods respectively.
    \cite{Viguerie2022} claims a computational speedup of 2.44.
    \cite{Puso2023} claims a speedup of at least a factor of 20
    using the multirate time integration described in \cite{Hodge2021}.
\end{itemize}

This work proposes a Domain Decomposition approach which poses the
system in the vecinity of the Heat Affected Zone (HAZ) in the
reference frame of the heat source, and the usual system is used
in the rest of the domain. Since the moving heat source is the main driver
of the process, it is plausible that most unknowns are
better expressed and less transient in its coordinate frame in the region close to it.
Moreover, multiple authors in the welding litterature
have observed that steady state is reached rather quickly in the heating track
\cite{Mundra1996, VanElsen2007, Powar2016}. In this work,
the steadiness of the moving subproblem is taken advantage off 
in order to take bigger and bigger timesteps.\par

Although this work focuses on the benefits of this framework
when tackling the temporal multiscale nature of AM,
the increased accuracy of the results and benefits
when tackling the spatial multiscale nature of AM \cite{Storti2022, carraturo2021twolevel}
are also touched upon.\par
